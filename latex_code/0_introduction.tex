
This is a brief summary of the course TMA4267 about linear statistical models. It includes the main content from the lecture held by ... recorded in, where some examples etc... are excluded. 

The purpose of the notes is to give a good overview of the syllabus. I intend to add summaries of the lectures as I review them. I hope to include insights from projects / exercises where it is appropriate. 

\subsection*{Topics}
The first chapter begins by introducing multivariate distributions and how to compute expectations. We then move on to multivariate moments and transformations. Principal component analysis (PCA) is described, before we explain the need for charactestic functions and not just moment generating functions when working with multivariate distributions. 

In the second chapter we introduce the multivariate normal distribution. We deal with estimation in the multivariate normal distribution and give the theory of quadratic forms and idempotent matrices.

The third chapter tackles multiple linear regression. We introduce the model and its assumptions and estimate the parameters. Properties of the estimators, fitted values and residuals are established, and then put to use in performing inference about the coefficients. We perform t-tests, do ANOVA, compute the coefficient of determination and perform general F-tests. Finally, we look at some way of transforming the data.

In the fourth chapter, we analyse the model and the selection of models. We perform multiple hypothesis testing and present examples. 

In the fifth and final chapter, we do more ANOVA and investigate design of experiment. Our focus is two-level factorial design.