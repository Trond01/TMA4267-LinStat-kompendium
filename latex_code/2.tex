\section{The Multivariate Normal Distribution}
%%%%%%%%%%%%%%% Lecture 7 %%%%%%%%%%%%%%%
\subsection*{Lecture 7}

\subsection{Estimation of the multivariate normal distribution}
\subsubsection{Univariate case}
From the univatiate case we recall that if $X_1,\dots,X_n \sim N(\mu, {\s^2})$ are independent, then the MLE estimators are:
\begin{align*}
    \mh &= \bar{X} = \frac{1}{n} \sum_{i=1}^n X_i \\
    \s^2 &= \frac{1}{n} \sum_{i=1}^n (X_i-\bar{X})^2
\end{align*}
But since we found that $\ev{\s^2}=\frac{n-1}{n} \s^2$ is biased, we use instead the estimator
$$
    S^2 = \frac{1}{n-1} \sum_{i=1}^n (X_i-\bar{X})^2.
$$
We further proved that
\begin{enumerate}
    \item $\bar{X} \sim N(\mu, \s^2/n)$
    \item $\frac{(n-1)S^2}{\s^2} \sim \chi^2_{n-1}$
    \item $\bar{X}, S^2$ are independent (for the normal distribution)
    \item $\sqrt{n} \frac{\bar{X}-\mu}{S} \sim t_{n-1}$ (student t distribution)
\end{enumerate}
Our next goal is to obtain the result for the multivariate case. 
\subsubsection{Multivatiate case}
In this case, we have $p$-variate independent randum vectors $\X_1,...,\X_n \sim N(\M, \SS)$ where we denote $\X_i=(X_{i1},\dots,X_{ip})^T$. These make up the columns of the ($n\times p$) \term{data matrix} or \term{feature matrix} $\X$ given as:
$$
    \X = 
    \underset{\mr{features} \rightarrow}{
    \begin{pmatrix}
        X_{11} & \cdots & X_{1p} \\
        \vdots & \ddots & \vdots \\
        X_{n1} & \cdots & X_{np}
    \end{pmatrix}}
    \substack{\text{samples} \\ \downarrow}
    = \begin{pmatrix}
        \X_1^T \\ \vdots \\ \X_n^T
    \end{pmatrix}= (\X_1 \cdots \X_n)^T.
$$
We want to estimate $\M, \SS$. Again we denote:
\begin{align*}
    \Mh &= \bar{\X} = \frac{1}{n} \sum_{i=1}^n \X_i = \frac{1}{n}\X^T \one \\
    \bs{S}^2 
    &= \frac{1}{n} \sum_{i=1}^n (\X_i-\bar{\X})^T(\X_i-\bar{\X}) 
    = \frac{1}{n} \X^T\brac{\mI-\frac{1}{n} \one\one^T}\X.
\end{align*}
The matrix $\mC=\mI-\frac{1}{n} \one\one^T$ is called the \term{centering matrix} because its action is to remove the mean of a vector:
$$
    \mC \y = \begin{pmatrix}
        1-\frac{1}{n} & \cdots & \frac{1}{n} \\
        \vdots & \ddots & \vdots \\
        \frac{1}{n} & \cdots & 1-\frac{1}{n}
    \end{pmatrix}
    \begin{pmatrix}
        y_1 \\ \vdots \\ y_n
    \end{pmatrix}
    = 
    \begin{pmatrix}
        y_1-\bar{y} \\ \vdots \\ y_n-\bar{y}
    \end{pmatrix}.
$$
We note that 
\begin{enumerate}
    \item $\mC$ is symmetric
    \item $\mC$ is idempotent
\end{enumerate}
For the estimators, we may prove:
\begin{proposition}
    $\bar{\X} \sim N(\M, \frac{1}{n}\SS)$
\end{proposition}
\begin{proof}
    By characteristic function.
\end{proof}
\begin{proposition}
    $\ev{\bs{S}} = \frac{n-1}{n} \SS.$
\end{proposition}
\begin{proof}
    
\end{proof}
Hence we obtain an unbiased estimator as
$$
    \bs{S} = \frac{1}{n-1} \sum_{i=1}^n (\X_i-\bar{\X})(\X_i-\bar{\X})^T.
$$

%%%%%%%%%%%%%%% Lecture 8 %%%%%%%%%%%%%%%
\subsection*{Lecture 8}

\subsubsection{Quadratic forms}
Let $\X=(X_1,\dots,X_p)^T$ be a ($p\times 1$) random vector and $\mA=(a_ij)\in\R^{p\times p}$ a matrix. This gives a \term{quadratic form}:
$$
    \X^T\mA\X = \sum_{i,j} X_ia_{ij}X_j.
$$ 
\begin{theorem} \term{(Trace formula)}
    $
        \ev{\X^T\mA\X} = \mr{tr}(\bs{A\Sigma}) + \M^T\mA\M
    $    
\end{theorem} 
\begin{proof}
    Using that $\cov{X_i,X_j} = \ev{X_iX_j}-\ev{X_i}\ev{X_j}$ we obtain the result:
    \begin{align*}
        \ev{\X^T\mA\X}
        &= \sum_{ij}a_{ij}\ev{X_iX_j}
        = \sum_{ij}a_{ij}(\Sigma_{ij}-\mu_i\mu_j)
        \\&= \sum_{ij}a_{ij}\Sigma_{ij} -  \sum_{ij}a_{ij}\mu_i\mu_j
        = \mr{tr}(\mA\SS) - \M^T\mA\M.
    \end{align*}
\end{proof}

\subsubsection{Idempotent matrices}
Recall that the (square) matrix $\mA$ is said to be idempotent of $AA=A$. We have the following:
\begin{proposition}
    Let $\mA$ be idempotent.
    \begin{enumerate}
        \item $\mI-\mA$ is also idempotent.
        \item $\mA(\mI-\mA)=(\mI-\mA)\mA=\zero$.
        \item The only non-singular idenpotent matrix is $\mI$.
        \item All eigenvalues of idempotent matrices are $0$ or $1$.
    \end{enumerate}
\end{proposition}
\begin{proof}
    We have:
    \begin{enumerate}
        \item $(\mI-\mA)^2=\mI^2-\mA\mI-\mI\mA+\mA^2=0$.
        \item Obvious
        \item Later
        \item $\lambda\x=\mA\x=\mA\mA\x=\lambda^2\x$.
    \end{enumerate}
\end{proof}
Before our next important result, we recall from linear algebra that:
\begin{enumerate}
    \item Let $\mA\in\R^{p\times p}$ have eigenvalues $\lambda_1,\dots,\lambda_p$. Then:
        \begin{align*}
            \mr{tr}(\mA) = \sum \lambda_i, \quad
            \mr{det}(\mA) = \prod \lambda_i.
        \end{align*}
    \item The matrix $\mA\in\R^{p\times p}$ of rank $r$ has \term{spectral decomposition}:
        $$
            \mA = \bs{P}\mL\bs{P}^T
        $$
        where $P$ is the matrix with eigenvalues as columns and $\mL$ the diagonal matrix of eigenvalues. Note that $\bs{P}\in\R^{p\times r}, \mL\in\R^{r\times r}$.
        \TODO{something about orthonormal}
\end{enumerate}
Note that for $\mA$ symmetric and idempotent, we have $\mL=\mI_r$ in point 2 and $\mr{rank}(\mA)=\mr{tr}(\mA)=r$ due to point 1. 

The following theorem will be used extensivelly in the sequel. 
\begin{theorem}
    $\bs{Z}\sim N(0, \mI_p)$ and let $\bs{R}, \bs{S}$ be symmetric and idempotent of rank $r, s$ respectivelly. Suppose also $\bs{RS}=0$. Then 
    \begin{enumerate}
        \item $\bs{Z}^T \bs{R} \bs{Z} \sim \chi^2_r$
        \item $\bs{Z}^T \bs{R} \bs{Z}$ and $\bs{Z}^T \bs{S} \bs{Z}$ are independent
        \item $\frac{\bs{Z}^T \bs{R} \bs{Z} / r}{\bs{Z}^T \bs{S} \bs{Z}/s} \sim F_{r,s}$.
    \end{enumerate}
\end{theorem}
\begin{proof} We have: \TODO{
    \begin{enumerate}
        \item 
        \item 
        \item By definition.
    \end{enumerate}}
\end{proof}