\section{Multiple Linear Regression}
We assume that we have $i=1,\dots,n$ observations of a \term{response variable} $Y_i$ depending on $k$ \term{explanatory variables} $x_{ij}$ through a linear model:
$$
    Y_i = \b_0 + \b_1 x_{i1} + \dots + \b_k x_{ik} + \e_i.
$$
It can be written on matrix form as:
$$
    \underset{\Y}{
    \begin{pmatrix}
        Y_1 \\
        Y_2 \\
        \vdots \\
        Y_n
    \end{pmatrix}}
    =
    \underset{\X}{
    \begin{pmatrix}
        1 & x_{11} & x_{12} & \cdots & x_{1k} \\
        1 & x_{21} & x_{22} & \cdots & x_{2k} \\
        \vdots & \vdots & \vdots & \ddots & \vdots \\
        1 & x_{n1} & x_{n2} & \cdots & x_{nk}
    \end{pmatrix}}
    \underset{\B}{
    \begin{pmatrix}
        \beta_0 \\
        \beta_1 \\
        \vdots \\
        \beta_k
    \end{pmatrix}}
    +
    \underset{\E}{
    \begin{pmatrix}
        \varepsilon_1 \\
        \varepsilon_2 \\
        \vdots \\
        \varepsilon_n
    \end{pmatrix}}.
$$
The matrix $\X$ is reffered to as the \term{design matrix}. The $\e$'s are \term{errors} and the $\b$'s the \term{parameters}.

%%%%%%%%%%%%%%% Lecture 9 %%%%%%%%%%%%%%%
\subsection*{Lecture 9}

Assumptions
\begin{enumerate}
    \item $\X$ is of cull column rank
    \item $E\E=\boldsymbol{0}$
    \item Homostochastic: $\var(\e_i) = 0 \quad\forall i$.
    \item If $\X$ is random, then 2 and 3 are conditioned on $\X$. 
    \item Normality of errors: $\E\sim N(0, \s^2I_n)$.
\end{enumerate}

... obtain least squares estimators $\Bh, \Sh^2$ of $\B, \S^2$

Residuals ...

\subsubsection*{Parameter estimation}

Two approaches: LSE and MLE ...

$$
    \Bh = \argmin_{\B\in\R^{k+1}} \sum_{i=1}^n (Y_i - \x_i^T\B)^2
$$

... deducing that LSE and MLE give the same result ...

...

Hat matrix

ghjfiodeifgoerjfkdworw9u0gryhj

Du fulgte ikke med nei

%%%%%%%%%%%%%%% Lecture 10 %%%%%%%%%%%%%%%
\subsection*{Lecture 10}



%%%%%%%%%%%%%%% Lecture 11 %%%%%%%%%%%%%%%
\subsection*{Lecture 11}

\begin{figure}[H] \centering
    \begin{tikzpicture}[scale=1.5]
        % Define X
        \draw[-] (0,0) -- (2,1) node[right] {$\rm{Col}(\X)$};
        % Define Y
        \draw[-] (0.6,0.3) node[below]{$\Y'$} -- (1,2) node[above] {$\Y$};
        % Projection
        \draw[-] (1,2) -- (1.6,0.8) node[below]{$\hat{\Y}$};
    \end{tikzpicture}        
\end{figure}



%%%%%%%%%%%%%%% Lecture 12 %%%%%%%%%%%%%%%
\subsection*{Lecture 12}

questions about independence. Detour into sigma algebras etc ...

\begin{theorem}
    Suppose $X, Y$ are independent random variables and that $f, g$ are two measurable functions. Then $f(X), g(Y)$ are also independent. 
\end{theorem}

ANOVA - Analysis of variance

\begin{theorem} \term{(ANOVA decomposition)}\label{thm:ANOVA}
    Assuming the necesarry assumptions, 
    $$
    \underset{\mr{SST}}{\underbrace{
        \sum_{i=1}^n (Y_i - \bar{Y})}} = 
    \underset{\mr{SSR}}{\underbrace{
        \sum_{i=1}^n (\hat{Y}_i - \bar{Y})}} + 
    \underset{\mr{SSE}}{\underbrace{
        \sum_{i=1}^n (Y_i - \hat{Y}_i)^2}}. 
    $$
\end{theorem}
\begin{proof}
    \TODO{there aint space in the margin}
\end{proof}
The 3 sums are called \term{total sum of squares}, \term{regression sum of squares} and \term{error sum of squares} respectivelly. 
This decomposition motivates the following definition. 
\begin{definition}
    The part of the total variation due to the model is called the \term{coefficient of determination} or the \term{R2-score}:
    $$
        R^2 = \frac{\mr{SSR}}{\mr{SST}} \overset{\mr{thm}} = 1 - \frac{\mr{SSE}}{\mr{SST}}.
    $$
\end{definition}
One may also prove another representation:
$$
    R^2 = \frac{\brac{\sum_{i=1}^n(Y_i-\bar{Y})(\hat{Y}_i-\bar{Y})}^2}{\sum_{i=1}^n(Y_i-\bar{Y})^2\sum_{i=1}^n(\hat{Y}_i-\bar{Y})}.
$$
This is the square of the empirical correlation between $\Y, \hat{\Y}$.

%%%%%%%%%%%%%%% Lecture 13 %%%%%%%%%%%%%%%
\subsection*{Lecture 13}


\subsubsection*{Fictional model}
"Fictional model" using $x_{ij}$ as response for some fixed feature $j$. 

\vspace*{2.5em}
\[
    y = \EqM{c1}[red]{c_1}p\EqM{p1}[green!50!black]{^{1}} + \EqM{c2}{c_2}p^2 + \ldots{} + \EqM{cn}{c_n}p^{n} + \EqM{cn1}{c_{n+1}}x^{n+1}
\]
\vspace*{2.5em}
\begin{tikzpicture}[
    remember picture, 
    overlay,
    tips=proper,
    every node/.style={draw},
]
    \draw[red,<-] 
        ++(c1.north) 
        -- ++(0,2em) 
        -- ++(-1em,0) 
        node[anchor=east] {First scalar}; 
        
    \draw[green!50!black,<-] 
        ++(p1.north) 
        -- ++(0,2em) 
        -- ++(1em,0) 
        node[anchor=west] {Lowest exponent}; 
    
    \draw[blue] 
        ++(c1.south) 
        ++(0, -2em) 
        node[anchor=north west, outer sep=3pt] (ScalarLabel) {A bunch of different scalars};
    
    \draw[blue,->] 
        ++(ScalarLabel.170)  
        edge[in=215,out=45] 
        (c1.south);            
        
    \draw[blue,->] 
        ++(ScalarLabel.163)  
        edge[in=215,out=45] 
        (c2.south);            

    \draw[blue,->] 
        ++(ScalarLabel.19)  
        edge[in=-45,out=-215] 
        (cn.south);            

    \draw[blue,->] 
        ++(ScalarLabel.10)  
        edge[in=-45,out=-215] 
        (cn1.south);
\end{tikzpicture}




... \TODO

\subsubsection*{General F-test}

\TODO{important to have on yellow paper}

We set up a much more general problem. Let $A\in\R^{r\times p}$, $r<p$, $\mathrm{rank}(A)=r$, $\boldsymbol{d}\in\R^d$. We test the hypothesis:
$$
    \mr{H}_0 : A\B = \bs{d}, 
    \quad\quad\quad
    \mr{H}_1 : A\B \ne \bs{d}.
$$
Some special cases of this general setup are.
\begin{enumerate}
    \item $r=1, d=0, A=(0, \dots, 1, \dots, 0)$ with $1$ at index $i$, gives the test 
    $$
        \mr{H}_0 : \b_i = 0, 
        \quad\quad\quad
        \mr{H}_1 : \b_i \ne 0.
    $$
    \item $r=1, d=0, A=(0, \dots, 1, \dots, -1, \dots, 0)$ with $1$ at index $i$ and $-1$ at index $j$, gives the test 
    $$
        \mr{H}_0 : \b_i = \b_j,
        \quad\quad\quad
        \mr{H}_1 : \b_i \ne \b_j.
    $$
    \item $r=k, d=\bs{0}\in\R^k, A=(\bs{0}, \mr{diag}(1))\in\R^{k\times p}$, gives the test 
    $$
        \mr{H}_0 : \b_i = 0 \quad\forall i\in\{1,\dots,k\},
        \quad\quad\quad
        \mr{H}_1 : \b_i \ne 0 \textrm{ for some } i\in\{1,\dots,k\}.
    $$
\end{enumerate}

%%%%%%%%%%%%%%% Lecture 14 %%%%%%%%%%%%%%%
\subsection*{Lecture 14}

Let $\mathcal{B}$ be the space of $\B$ satisfying $\mr{H}_0$. The restricted problem is:
$$
    \Bh^R = \argmin_{\B\in\mathcal{B}}(\Y-\X\B)^T(\Y-\X\B).
$$
Using lagrange multipliers and a bag of tricks, we obtain:
$$
    \Bh^R = \Bh - (\X^T\X)^{-1}\bs{A}^T(\bs{A}(\X^T\X)^{-1}\mA^T)^{-1}(\bs{A}\Bh-\bs{d}).
$$
Denoting $\Delta = \Bh-\Bh^R$, we find:
$$
    \mr{SSE}^R = \mr{SSE} + \Delta^T\X^T\X\Delta
$$

... IMPORTANT: the concrete expressions for the F statistic...

We claim that the under $\mr{H}_0$, we have

$$
    F = \frac{\mr{SSE}^R-\mr{SSE} / r}{\mr{SSE}/(n-p)} \sim F_{r, n-p}.
$$

\begin{proof}
    
    what the

\end{proof}




%%%%%%%%%%%%%%% Lecture 15 %%%%%%%%%%%%%%%
\subsection*{Lecture 15}


... example ...


\subsubsection*{Transformations of data}
Motivation:
...


box cox transformation
 

variance stabilising transformation

Suppose $\mu=\E(Y_i)$ and that $\var(Y_i)$ depends on $\mu$. ...
